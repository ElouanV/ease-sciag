\section{Introduction}\label{sec:intro}
To work with \LaTeX, \href{https://fr.overleaf.com/}{Overleaf} is the \textbf{best} tool. It allows you to edit and compile your project. You can even share your project with a mate in case of a team project, and edit it together. Overleaf also provides a lot of documentation and templates for any document: a homework report, an article, a resume etc.
In this template, I will will try to show  a lot of usefull packages and tips to create a clean report for your homework. This is a template, which is not perfect, feel free to modify it to fit to your need. If you find alternatives or want to improve this template/tutorial, do not hesitate to \href{https://github.com/ElouanV/ease-sciag}{contribute}. Once you get a little used to using latex to write your reports, you will be able to write and format you report fatser than if you use Microsoft Words, and there is a good chance that the rendering will be prettier and more professional. Let the Office 365 suite to the marketing team.
Feel free to modify everything in \textit{main.tex} to understand how each command impact the template.

As you do not code a whole IT project in a single file, we do not write all our report in \textit{.tex} one file. Here we split it into one file for each \textit{section} of our report, and \textit{main.tex} use the \textsc{input} to include each file to the project.

If you use this template for a project, do not forget to replace all default value as \textsc{AUTHOR NAME}. You can also add or remove section according to the teacher expectations, this is just a skeleton and it will not answer to each project needs.

A lot of packages allow some configuration, for example, the \textit{hyperref} package allows to add color or not to hyperlinks. You can check the \textit{main.tex} file.
If in  a report you add a reference to a paper, you can use \href{https://fr.overleaf.com/learn/latex/Bibliography_management_with_bibtex}{BibTeX} to add a reference to it. In \textit{main.tex}, you can see a way to configure and include your .bib file to your report. You can easily find the bibtex format of a paper citation on the web site you used to find it. (On \href{https://scholar.google.com/}{Google Scholar}, there is a "Quote" button when you are searching for a paper, and you can select the bibtex format). Once your paper is in your \textit{.bib} file, you can cite a paper like this \cite{zhang2022explaining}.