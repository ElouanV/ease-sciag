\section{Method}
You can add references to your annexes like this Fig. \ref{fig:sciag_logo}.
Note that with Python, you can easily export plots in a vectorial format so that it scales well on your pdf. You can also easily convert a data frame to a \textit{.tex} file to avoid copy/paste error and avoid wasting time.
Some libraries like \textit{minted} or \textit{listings} allows you to add code in your project and format it well. Its avoid to include a screen shot of your code in your report which is pretty ugly and annoying when you edit one line in your code...
If you need to include an algorithm in your report, you can use \textit{algorithm} or \textit{algpseudocode}.
Here are some examples of libraries we just talk: algorithm \ref{alg:median_approx} presents an algorithm, Fig. \ref{code:fibo} show a Python code.

%% Median graph approximation algorithms
\begin{algorithm}[htb]\label{alg:random_alg}
\caption{Approximate median graph}
\label{alg:median_approx}
\begin{algorithmic}[1]
\Require $G$ a set of graphs, $t$ a threshold
\Ensure Median graph
\State S $\leftarrow  \{\}$ 
\State g $\leftarrow$ drawn from $G$
\State S $\leftarrow S \cup \{g\}$\;
\Repeat %While{$dist(g, S) > t$}
\State     draw $g \sim dist(g,S)$ \;
\State     S $ \leftarrow$ S $\cup$ $\{g\}$;
\Until $(dist(g, S) < t)$
\State \Return median(S)\;
\end{algorithmic}
\end{algorithm}

\begin{figure}[!htb]
    \centering
    \lstinputlisting[language=Python]{code/fibo.py}
    \caption{Recursive implementation of fibo in Python}
    \label{code:fibo}
\end{figure}
Here is a SPARQL request with \textit{minted}
\begin{minted}{sparql}
select (count(*) as ?count) where {
  ?x ?p ?y
}
\end{minted}


With the \$ symbol, you can write mathematical sentences: $\frac{n!}{k!(n-k)!} = \binom{n}{k}$.But you can also write equation like this: 
\begin{equation}
    \mathcal{}
\end{equation}
There is a lot of ways to add maths to a report in LateX, RTFM\footnote{You can add footnote like this, and here is a link to the overleaf documentation for \href{https://www.overleaf.com/learn/latex/Mathematical_expressions}{mathematical expression}}. This also shows you how to add an hyperlink to your report.\\
You can also add a table as in Tab. \ref{tab:tinytab}.

\begin{table}[htb]
    \centering
    \begin{tabular}{| c | c |}
    \hline
         ColName & ColName  \\
         \hline
         Raw1 & Raw1 \\
         \hline
    \end{tabular}
    \caption{Add a caption }
    \label{tab:tinytab}
\end{table}

Note than you can add a label to almost everything to add a reference to, like this reference to the Section \ref{sec:intro}.